% !TEX root = simfempy.tex
%

%----------------------------------------
%
\usepackage{imakeidx}
\makeindex[columns=2, title=Index alphabétique, intoc]

\usepackage{ifluatex}
%
\usepackage{amsmath,amssymb,amsthm,latexsym}
% default !!
\usepackage[hmargin=2.5cm, top=3.cm, bottom=3.cm]{geometry}
\usepackage{stmaryrd}
%
%---------------------------------------------
\ifluatex
\usepackage{csquotes}
\usepackage[english]{babel}
\usepackage[no-math]{fontspec}
\setmainfont{Palatino}
\usepackage[OT1,euler-digits]{eulervm}
%---------------------------------------------
\else
\usepackage[utf8]{inputenc}
\usepackage[T1]{fontenc} % un second package
\usepackage[english]{babel}
\usepackage{eulervm,palatino}
\fi
%---------------------------------------------
%
\usepackage{hyperref}
\usepackage{url}
\usepackage[english,algoruled,lined]{algorithm2e}
\usepackage{listings}
\usepackage{xcolor}
\usepackage[many]{tcolorbox}
\tcbuselibrary{breakable}
\usepackage{multirow}
%
\usepackage[backend=bibtex,sorting=none,firstinits=true, style=numeric, doi=false,isbn=false,url=false]{biblatex}
\bibliography{../../Bibliotheque/bibliotheque.bib}

%\usepackage{makeidx}
%\makeindex

%---------------------------------------------
\usepackage[toc,page]{appendix}
\usepackage{arydshln}
%
\renewcommand{\thefigure}{\thesection.\arabic{figure}}
\numberwithin{equation}{section}



%---------------------------------------------------------
\newtheorem*{theorem*}{Theorem}
\newtheorem*{lemma*}{Lemma}
\newtheorem*{definition*}{Definition}
\newtheorem*{remark*}{Remark}
\newtheorem*{corollary*}{Corollary}
\newtheorem*{example*}{Example}
\newtheorem{theorem}{Theorem}[section]
\newtheorem{definition}[theorem]{Definition}
\newtheorem{proposition}[theorem]{Proposition}
\newtheorem{corollary}[theorem]{Corollary}
\newtheorem{lemma}[theorem]{Lemma}
\newtheorem{remark}[theorem]{Remark}
\newtheorem{example}[theorem]{Example}


%----------------------------------------
\usepackage{stmaryrd}
\usepackage{etoolbox}
\usepackage{xcolor}
\usepackage{cancel}
\usepackage{mathtools} 
\usepackage{enumitem}
\setlist[enumerate,2]{label=\arabic*)}
% label* keeps label for lower level
\setlist[enumerate]{label*=\Roman*)}
\setlist[enumerate,1]{label=\alph*)}
%\setlist[itemize,1]{label=\textbullet}
%\setlist[itemize,2]{label=$\triangleright$}
%----------------------------------------
%
\newenvironment{yellow}{\begin{tcolorbox}[boxrule=2pt, colback=yellow!10!white]}{\end{tcolorbox}}
\definecolor{myblue}{rgb}{0.004, 0.1, 0.57}
\definecolor{myred}{rgb}{0.58, 0.066, 0.}
\definecolor{mygreen}{rgb}{0.24, 0.55, 0.15}
\definecolor{mygray}{rgb}{0.28, 0.28, 0.28}
\definecolor{myviolet}{rgb}{0.3, 0.1, 0.4}
\newcommand{\blue}[1]{\textcolor{myblue}{#1}}
\newcommand{\red}[1]{\textcolor{myred}{#1}}
\newcommand{\green}[1]{\textcolor{mygreen}{#1}}
\newcommand{\gray}[1]{\textcolor{mygray}{#1}}
\newcommand{\violet}[1]{\textcolor{myviolet}{#1}}
%
\newcommand*\circled[1]{\tikz[baseline=(char.base)]{
  \node[shape=circle,draw,inner sep=1pt] (char) {#1};}}

\newcommand{\commentaire}[1]{\blue{#1}}
%
\newcommand\addpicture[4]
{%
\begin{wrapfigure}{r}{#2\textwidth}
  \begin{center}
    \includegraphics[width=#2\textwidth]{#1}
  \end{center}
  \caption{\label{#4}#3}
\end{wrapfigure} 
}
%----------------------------------------
% Spaces
\newcommand\TildeHdiv[1][\Omega]{\widetilde{H}_{\rm div}(#1)}
\newcommand\Hdiv[1][\Omega]{H_{\rm div}(#1)}
%\newcommand\Hdiv[1][]{\ifstrempty{#1}{H({\rm div},\Omega)}{H({\rm div},#1)}}
%\newcommand{\Hdivzero}{H_0({\rm div},\Omega)}
\newcommand\Hcurl[1]{H_{\rm curl}(#1)}
\newcommand{\Pkhom}[1]{P_{\rm hom}^{#1}}
\newcommand{\Phom}{\Pkhom{k}}

\newcommand{\multin}{\mathcal I_n}
%----------------------------------------
% Divers
\newcommand{\spacenn}[1]{#1\setminus\Set{0}} 
\newcommand{\lambdamin}{\lambda_{\rm min}} 
\newcommand{\lambdamax}{\lambda_{\rm max}} 
\newcommand{\inti}[2]{\llbracket #1, #2 \rrbracket} 
\newcommand{\subscr}[2]{#1_{\rm #2}} 
\newcommand{\argmin}{\operatorname{argmin}} 


% AFEM
%
%\newcommand{\REF}{{\mathtt{REF}}}
\newcommand{\MARK}{{\mathtt{MARK}}}
\newcommand{\meshcriterion}{\mathcal C}
\newcommand{\mcM}{\mathcal M}
%
%
\newcommand{\Cmesh}{C_{\rm mesh}}
\newcommand{\Cglobrel}{C_{\rm gr}}
\newcommand{\Cloceff}{C_{\rm le}}
\newcommand{\Copt}{C_{\rm opt}}
\newcommand{\Cneighb}{C_{\rm nei}}
\newcommand{\Cmeshopt}{C_{\rm mopt}}
\newcommand{\Cgeom}{C_{\rm g}}
\newcommand{\Cmeshcrit}{C_{\rm mc}}
\newcommand{\Cmon}{C_{\rm mon}}
\newcommand{\Cemon}{C_{\rm em}}
\newcommand{\Ceststab}{C_{\rm es}}
\newcommand{\Cinv}{C_{\rm inv}}
\newcommand{\qmesh}{q_{\rm m}}
\newcommand{\qsolve}{q_{\rm s}}

\newcommand{\chistop}[1]{{\chi_{#1}^{\rm fin}}}
\newcommand{\DeltaStop}[1]{{\widetilde{\Delta}_{#1}}}
\newcommand{\solvecrit}[1]{{\mathcal S_{#1}}}
\newcommand{\qred}{q_{\rm red}}
\newcommand{\Crel}{C_{\rm rel}}
\newcommand{\Cstab}{C_{\rm stab}}

\newcommand{\Cupp}{C_{\rm up}}
\newcommand{\Clow}{C_{\rm low}}
\newcommand{\true}{{\mathtt{true}}}
\newcommand{\false}{{\mathtt{false}}}
%
%----------------------------------------
% FEM
\newcommand{\ind}{{\tt ind}}
\newcommand{\hK}[1][K]{d_{#1}} 
\newcommand{\h}{d_h} 
\newcommand{\Cspace}{\mathcal{C}}
\newcommand{\Dspace}{\mathcal{D}}
\newcommand{\Pspace}{\mathcal{P}}
\newcommand{\Qspace}{\mathcal{Q}}
\newcommand{\CR}{\mathcal{C\!R}}
\newcommand{\RT}{\mathcal{R\!T}}
\newcommand{\BDM}{\mathcal{B\!D\!M}}
\newcommand{\Ned}{N}


\newcommand{\allmeshes}{\mathcal H}
%\newcommand{\allmeshes}{\mathcal H(\Omega)}
\newcommand{\Edges}{\mathcal E}
\newcommand{\Sides}{\mathcal S}
\newcommand{\Cells}{\mathcal K}
\newcommand{\Nodes}{\mathcal N}
\newcommand{\NodesInt}{{\mathcal N}^{int}}
\newcommand{\level}{\operatorname{lev}}
\newcommand{\SidesInt}{\mathcal S^{\rm int}}
\newcommand{\SidesBdry}{\mathcal S^{\partial}}
\newcommand{\meanS}[1]{\left\{\{#1\right\}_{S}}
\newcommand{\jumpS}[1]{\left[#1\right]_{S}}
\newcommand{\jump}[1]{\left[#1\right]}
\newcommand{\mean}[1]{\left\{#1\right\}}


%----------------------------------------
\newcommand{\R}{\mathbb R}
\newcommand{\N}{\mathbb N}
\newcommand{\C}{\mathbb C}
%\newcommand{\Rest}[2]{{#1}_{|_{#2}}}
\newcommand\Rest[2]{{% we make the whole thing an ordinary symbol
  \left.\kern-\nulldelimiterspace % automatically resize the bar with \right
  #1 % the function
  \vphantom{\big|} % pretend it's a little taller at normal size
  \right|_{#2} % this is the delimiter
  }}%
\newcommand{\normal}{\mathbf{\overrightarrow{n}}} 
\newcommand{\binormal}{\mathbf{\overrightarrow{b}}} 
\newcommand{\tangent}{\mathbf{\overrightarrow{t}}} 
%
\newcommand{\curl}{\operatorname{curl}} 
\newcommand{\grad}{\operatorname{grad}} 
\renewcommand{\div}{\operatorname{div}}
\newcommand{\Grad}{\operatorname{Grad}} 
\newcommand{\rot}{\operatorname{rot}} 
\newcommand{\Rot}{\operatorname{Rot}} 
\newcommand{\Div}{\operatorname{Div}} 
\newcommand{\dpfrac}[3][\partial]{\frac{#1 #2}{#1 #3}} 
\newcommand{\dn}[1]{\dpfrac{#1}{n}} 
\newcommand{\dbeta}[1]{\dpfrac{#1}{\beta}} 
\newcommand{\ddpfrac}[3]{\frac{\partial^2 #1}{\partial #2\partial #3}} 
%
\newcommand{\dt}[1]{\frac{d #1}{dt}}
\newcommand{\ddt}[1]{\frac{d^2 #1}{dt^2}}
\newcommand{\ds}[1]{\frac{d #1}{ds}}
\newcommand{\dds}[1]{\frac{d^2 #1}{ds^2}}
\newcommand{\ddds}[1]{\frac{d^3 #1}{ds^3}}
%
\newcommand{\diag}{\operatorname{diag}}
\newcommand{\detsans}[2]{\det\left(#1 \left\|#2\right.\right)}
%
\newcommand{\abs}[1]{\left|#1\right|} 
%
\newcommand{\In}[1]{{#1}^{\rm in}} 
\newcommand{\Ex}[1]{{#1}^{\rm ex}} 
\newcommand{\InEx}[1]{{#1}^{\rm in/ex}} 
\newcommand{\refer}[1]{{#1}_{\rm ref}} 
%\newcommand{\Ref}[1]{#1_{\rm ref}} 
\newcommand{\Set}[1]{\left\{#1\right\}} 
\newcommand{\SetDef}[2]{\left\{#1\;\middle|\;#2\right\}} 
\newcommand{\transpose}[1]{{#1}^{\mathsf{T}}} 
\newcommand{\transposeInv}[1]{{#1}^{\mathsf{-T}}} 
\newcommand{\eps}{\varepsilon}
\newcommand{\norm}[1]{\left\|#1\right\|}
\makeatletter
\newcommand{\opnorm}{\@ifstar\@opnorms\@opnorm}
\newcommand{\@opnorms}[1]{%
  \left|\mkern-1.5mu\left|\mkern-1.5mu\left|
   #1
  \right|\mkern-1.5mu\right|\mkern-1.5mu\right|
}
\newcommand{\@opnorm}[2][]{%
  \mathopen{#1|\mkern-1.5mu#1|\mkern-1.5mu#1|}
  #2
  \mathclose{#1|\mkern-1.5mu#1|\mkern-1.5mu#1|}
}
\makeatother
\newcommand{\tnorm}[1]{\opnorm{#1}}
%\newcommand{\tnorm}[1]{\left\|| #1\right|\|}
\newcommand{\normF}[1]{\tnorm{#1}_{\rm F}}
\newcommand{\scp}[2]{\left\langle#1,#2\right\rangle}
%
\newcommand{\vect}[1]{\operatorname{Vect}(#1)}
\newcommand{\supp}[1]{\operatorname{supp}(#1)}
\newcommand{\diam}{\operatorname{diam}}
%\newcommand{\conv}[1]{\operatorname{conv}\left(#1\right)}
\newcommand{\conv}{\operatorname{conv}}
\newcommand{\tr}{\operatorname{tr}}
\newcommand{\Id}{\operatorname{Id}}
%
\newcommand{\meshcup}{\dot{\cup}}
\newcommand{\meshcap}{\dot{\cap}}
\newcommand{\meshle}[2]{#1{\le}#2}
%
\newcommand{\sides}{\mathcal S}
\newcommand{\sidesint}{\mathcal S^{int}}
\newcommand{\sidesbound}{\mathcal S^{\partial}}
\newcommand{\cells}{\mathcal K}
\newcommand{\nodes}{\mathcal N}
\newcommand{\hmax}{h_{{max}}}
%
%
\newcommand{\sign}[1]{\operatorname{sgn}(#1)}
\newcommand{\esssup}{\operatorname{esssup}}
%
\newcommand{\northo}{{n^{\perp}}}
\newcommand{\uD}{u^{\rm D}}
\newcommand{\uDir}{u^{\rm D}}
\newcommand{\udir}{u^{\rm D}}
\newcommand{\uNeu}{u^{\rm N}}
\newcommand{\GammaD}{\Gamma_{\rm D}}
\newcommand{\GammaN}{\Gamma_{\rm N}}
\newcommand{\GammaR}{\Gamma_{\rm R}}
\newcommand{\vn}{v^{\rm R}_n}
\newcommand{\gR}{g^{\rm R}}
\newcommand{\TD}{T^{\rm D}}
\newcommand{\qN}{q^{\rm N}}
\newcommand{\qR}{q^{\rm R}}
\newcommand{\pN}{p^{\rm N}}
\newcommand{\vD}{v^{\rm D}}





%%----------------------------------------------
%\usepackage[many,breakable,skins]{tcolorbox}
%\usepackage{marginnote}
%\usepackage{xcolor}
%\usepackage[a4paper,margin=2.4cm]{geometry}
%\usepackage{amsmath,amssymb,amsthm,latexsym}
%%\usepackage[top=Bcm, bottom=Hcm, outer=Ccm, inner=Acm, heightrounded, marginparwidth=Ecm, marginparsep=Dcm]{geometry}
%%
%%----------------------------------------------
%\usepackage[utf8]{inputenc}
%\usepackage[T1]{fontenc}
%\usepackage[english]{babel}
%\usepackage{eulervm,palatino}
%%----------------------------------------------
%%
%\usepackage{hyperref}
%\usepackage{url}
%\usepackage[french,algoruled,lined]{algorithm2e}
%\usepackage{listings}
%\usepackage[many,breakable]{tcolorbox}
%\usepackage{cancel}
%\usepackage[toc,page]{appendix}
%\usepackage{stmaryrd}
%%---------------------------------------------------------
%\newcommand{\margincomment}[2][0cm]{\marginnote{\textcolor{blue}{#2}}[#1]}
%\newcommand*\circled[1]{\tikz[baseline=(char.base)]{
%  \node[shape=circle,draw,inner sep=1pt] (char) {#1};}}
%%---------------------------------------------------------
%\usepackage{thmtools}
%\renewcommand\thmcontinues[1]{continued}
%\declaretheorem[style=definition,numberwithin=section]{example}
%\declaretheorem[style=remark,numberwithin=section]{remark}
%\newtheorem{definition}{Definition}[section]
%\newtheorem{proposition}{Proposition}[section]
%\newtheorem{lemma}{Lemma}[section]
%\newtheorem{theorem}{Theorem}[section]
%%\newtheorem{remark}{Remark}[section]
%\newtheorem{corollary}{Corollary}[section]
%\numberwithin{equation}{section}
%\renewcommand{\thefigure}{\thesection.\arabic{figure}}
%%---------------------------------------------------------
%\newenvironment{uglyproof}{\begin{proof}\color{gray}}{\end{proof}}
%%
%%----------------------------------------
%\usepackage[backend=biber, bibencoding=utf8, defernumbers=true, refsection=section, giveninits=true, doi=false,isbn=false,url=false]{biblatex}
%\addbibresource{/Users/becker/Latex/Bibliotheque/bibliotheque.bib}
%
%%----------------------------------------
%\definecolor{myyellow}{rgb}{0.9, 0.9, 0.01}
%\definecolor{myblue}{rgb}{0.004, 0.1, 0.57}
%\definecolor{myred}{rgb}{0.58, 0.066, 0.}
%\definecolor{mygreen}{rgb}{0.24, 0.55, 0.15}
%\definecolor{mygray}{rgb}{0.28, 0.28, 0.28}
%\definecolor{myviolet}{rgb}{0.3, 0.1, 0.4}
\definecolor{myorange}{rgb}{0.4654205607476635, 0.33271028037383177, 0.20186915887850468}
%\newcommand{\blue}[1]{\textcolor{myblue}{#1}}
%\newcommand{\red}[1]{\textcolor{myred}{#1}}
%\newcommand{\green}[1]{\textcolor{mygreen}{#1}}
%\newcommand{\gray}[1]{\textcolor{mygray}{#1}}
%\newcommand{\violet}[1]{\textcolor{myviolet}{#1}}
%\definecolor{greenlight}{rgb}{0.95,1,0.95}
%\definecolor{bluelight}{rgb}{0.9,0.9,1}
%\definecolor{graylighy}{rgb}{0.975,0.975,0.975}
\definecolor{yellowlight}{rgb}{0.999, 0.999, 0.85}
\newenvironment{yellowbox}[1][]{\begin{tcolorbox}[title={#1} , boxrule=2pt, colback=yellowlight]}{\end{tcolorbox}}
%%-------PYTHON--------------------
%\usepackage{listings}
%%\usepackage{fontspec}
%%\newfontfamily\listingsfont[Scale=.7]{DejaVu Sans Mono}
%\lstset{ 
%  backgroundcolor=\color{graylighy},
%  basicstyle=\linespread{0.8}\ttfamily,
%%  basicstyle=\ttfamily\footnotesize,
%  breaklines=true,
%  commentstyle=\color{mygreen},
%  deletekeywords={...},
%  escapeinside={\%*}{*)},
%  frame=single,
%  showstringspaces=false,
%  language=Python,
%  keywordstyle = \color{myblue}
%}
\newcommand{\python}[1]{\textcolor{myblue}{\textbf{\tt{#1}}}}
%\newcommand{\pythonvar}[1]{\textcolor{mygreen}{\textbf{\tt{#1}}}}
%\newcommand{\pythonkey}[1]{\textcolor{myviolet}{\textbf{\tt{#1}}}}
\newcommand{\pythondef}[1]{\textcolor{myorange}{\textbf{\tt{#1}}}}
%%
%%----------------------------------------
%% Spaces
%\newcommand\TildeHdiv[1][\Omega]{\widetilde{H}_{\rm div}(#1)}
%\newcommand\Hdiv[1][\Omega]{H_{\rm div}(#1)}
%%\newcommand\Hdiv[1][]{\ifstrempty{#1}{H({\rm div},\Omega)}{H({\rm div},#1)}}
%%\newcommand{\Hdivzero}{H_0({\rm div},\Omega)}
%\newcommand\Hcurl[1]{H_{\rm curl}(#1)}
%\newcommand{\Pkhom}[1]{P_{\rm hom}^{#1}}
%\newcommand{\Phom}{\Pkhom{k}}
%\newcommand{\multin}{\mathcal I_n}
%%----------------------------------------
%% FEM
\newcommand{\xup}[1][K]{x_{#1}^{\sharp}}
%\newcommand{\xdown}[1][K]{x_{#1}^{\flat}}
%\newcommand{\ind}{{\tt ind}}
%\newcommand{\hK}[1][K]{d_{#1}} 
%\newcommand{\h}{d_h} 
%\newcommand{\Cspace}{\mathcal{C}}
%\newcommand{\Dspace}{\mathcal{D}}
%\newcommand{\Pspace}{\mathcal{P}}
%\newcommand{\Qspace}{\mathcal{Q}}
%\newcommand{\CR}{\mathcal{C\!R}}
%\newcommand{\RT}{\mathcal{R\!T}}
%\newcommand{\Ned}{N}
%\newcommand{\allmeshes}{\mathcal H}
%%\newcommand{\allmeshes}{\mathcal H(\Omega)}
%\newcommand{\Edges}{\mathcal E}
%\newcommand{\Sides}{\mathcal S}
%\newcommand{\Cells}{\mathcal K}
%\newcommand{\Nodes}{\mathcal N}
%\newcommand{\NodesInt}{{\mathcal N}^{int}}
%\newcommand{\level}{\operatorname{lev}}
%\newcommand{\SidesInt}{\mathcal S^{\rm int}}
%\newcommand{\SidesBdry}{\mathcal S^{\partial}}
%\newcommand{\meanS}[1]{\left\{\{#1\right\}_{S}}
%\newcommand{\jumpS}[1]{\left[#1\right]_{S}}
%\newcommand{\jump}[1]{\left[#1\right]}
%\newcommand{\mean}[1]{\left\{#1\right\}}
%%
%%----------------------------------------
%% meshes
%%
%\newcommand{\normal}[1][]{\vec{n}_{#1}}
%
%
%%----------------------------------------
%% parameters
%%
\newcommand{\kdiff}{k}
\newcommand{\kdiffinv}{k_{\rm inv}}
%
%%----------------------------------------
%% domain
%\newcommand{\GammaD}{\Gamma_{\rm D}}
%\newcommand{\GammaN}{\Gamma_{\rm N}}
%\newcommand{\GammaR}{\Gamma_{\rm R}}
%\newcommand{\udir}{u^{\rm D}}
%\newcommand{\uD}{u^{\rm D}}
%\newcommand{\uN}{u^{\rm N}}
%\newcommand{\uR}{u^{\rm R}}
%\newcommand{\pR}{p^{\rm R}}
%\newcommand{\pD}{p^{\rm D}}
%\newcommand{\vN}{v^{\rm N}}
%\newcommand{\vR}{v^{\rm R}}
\newcommand{\bdryint}[1]{{#1}^{\rm int}}
\newcommand{\bdrydir}[1]{{#1}^{\rm dir}}
\newcommand{\Vint}{\bdryint{V}}
\newcommand{\Vdir}{\bdrydir{V}}
\newcommand{\Aintdir}{{A}^{\rm int, dir}}
\newcommand{\Adirint}{{A}^{\rm dir, int}}
%%----------------------------------------
%% Language
%\newcommand{\pack}[1]{\textsl{#1}}
%%----------------------------------------
%% Optimization
%\newcommand{\Qad}{Q_{\rm ad}}
%\newcommand{\Qadh}{Q_{{\rm ad},h}}
%\newcommand{\Cd}{c_{\rm D}}
%\newcommand{\cD}{c^{\rm D}}
%%----------------------------------------
%% numbers
%\newcommand{\R}{\mathbb R}
%\newcommand{\N}{\mathbb N}
%\newcommand{\C}{\mathbb C}
%\newcommand{\Z}{\mathbb Z}
%%----------------------------------------
%% sets and functions
%\newcommand{\argmin}{\operatorname{argmin}}
%\newcommand{\Set}[1]{\left\{#1\right\}} 
%\newcommand{\SetDef}[2]{\left\{#1\;\middle|\;#2\right\}} 
%\newcommand{\vect}[1]{\operatorname{Vect}(#1)}
%\newcommand{\supp}[1]{\operatorname{supp}(#1)}
%\newcommand{\norm}[1]{\|#1\|}
%\newcommand{\eps}{\varepsilon}
%\newcommand{\scp}[2]{\left\langle#1,#2\right\rangle}
%\newcommand{\sgn}[1]{\operatorname{sgn}(#1)}
%\newcommand{\Rest}[2]{{#1}_{|_{#2}}}
%\makeatletter
%\newcommand{\opnorm}{\@ifstar\@opnorms\@opnorm}
%\newcommand{\@opnorms}[1]{%
%  \left|\mkern-1.5mu\left|\mkern-1.5mu\left|
%   #1
%  \right|\mkern-1.5mu\right|\mkern-1.5mu\right|
%}
%\newcommand{\@opnorm}[2][]{%
%  \mathopen{#1|\mkern-1.5mu#1|\mkern-1.5mu#1|}
%  #2
%  \mathclose{#1|\mkern-1.5mu#1|\mkern-1.5mu#1|}
%}
%\makeatother
%\newcommand{\tnorm}[1]{\opnorm{#1}}
%\newcommand{\abs}[1]{\left|#1\right|} 
%\newcommand{\conv}{\operatorname{conv}} 
%%----------------------------------------
%% diff
%\newcommand{\dpfrac}[3][\partial]{\frac{#1 #2}{#1 #3}} 
%\newcommand{\dn}[1]{\dpfrac{#1}{n}} 
%\newcommand{\dt}[1]{\dpfrac{#1}{t}} 
%\newcommand{\dx}[1]{\dpfrac{#1}{x}} 
%\newcommand{\ddt}[1]{\frac{d^2 #1}{dt^2}}
%\renewcommand{\div}{\operatorname{div}}
%\newcommand{\grad}{\operatorname{grad}}
%\newcommand{\gradS}{\operatorname{grad}_{\rm s}}
%\newcommand{\rot}{\operatorname{rot}}
%%----------------------------------------
%% linear algebra
%\newcommand{\transpose}[1]{{#1}^{\mathsf{T}}} 
%\newcommand{\transposeInv}[1]{{#1}^{\mathsf{-T}}} 
%\newcommand{\trace}{\operatorname{tr}} 
%\newcommand{\adj}{\operatorname{adj}} 
%\newcommand{\diag}{\operatorname{diag}}
