% !TEX root = ../simfempy.tex
%
%==========================================
\section{Mixed finite element methods}\label{sec:}
%==========================================
%
One frequently encounters equations of the form
%
\begin{equation}\label{eq:SaddlePointSystem}
\mathcal A \begin{bmatrix}x\\y\end{bmatrix}= \begin{bmatrix}f\\g\end{bmatrix},\quad \mathcal A=
\begin{bmatrix}
A & \transpose{B}\\
B & 0
\end{bmatrix}.
\end{equation}
%
For example, if $A$ is symmetric positive, such a system arises in the constrained minimization problem
%
\begin{align*}
\inf \Set{ E(x) :\; x\in V,\; Bx=g },\quad  E(x) := \frac12 \scp{A x}{x} ) -\scp{f}{x},
\end{align*}
%
since it expresses the stationarity of the Lagrangian
%
\begin{align*}
\mathcal L(x,y) = E(x)+ \scp{Bx-g}{y}.
\end{align*}

%
Under the assumption that $A$ is an isomorphism, we can use block-elimination to obtain a single equation in $y$. This can be done by the block-factorization
%
\begin{equation}\label{eq:SystemFactorized}
\mathcal A = \begin{bmatrix}
A & \transpose{B}\\
B & 0
\end{bmatrix}
= 
\begin{bmatrix}
I & 0\\
BA^{-1} & I
\end{bmatrix}
\begin{bmatrix}
A & 0\\
0 & S
\end{bmatrix}
\begin{bmatrix}
I & A^{-1}\transpose{B}\\
0 & I
\end{bmatrix},
\end{equation}
%
with the Schur complement
%
%
\begin{equation}\label{eq:Schur}
S := - BA^{-1}\transpose{B}.
\end{equation}
%
Now the question of regularity of the whole system $\mathcal A$ is equivalent to the regularity of $S$.

If $S$ is regular we then have
%
\begin{align*}
{\mathcal A}^{-1}
= 
\begin{bmatrix}
I & -A^{-1}\transpose{B}\\
0 & I
\end{bmatrix}
\begin{bmatrix}
A^{-1} & 0\\
0 & S^{-1}
\end{bmatrix}
\begin{bmatrix}
I & 0\\
-BA^{-1} & I
\end{bmatrix}
\end{align*}
%
%
%-------------------------------------------------------------------------
\subsection{Iterative solution}\label{subsec:}
%-------------------------------------------------------------------------
%
Using the factorization (\ref{eq:SystemFactorized}), the system (\ref{eq:SaddlePointSystem}) can be solved in three steps:
%
\begin{equation}\label{eq:}
%
\left\{
\begin{aligned}
&A \tilde x &&= f\\
&S y &&= g-B\tilde x\\
&A(x-\tilde x) &&=  - \transpose{B} y
\end{aligned}
\right.
%
\end{equation}
%
The first computes $\tilde x$, the solution neglecting the constraints. The last step computes a correction to $\tilde x$ in order to satisfy the constraint, requiring the multiplier $y$ to solve the second equation.
%
%-------------------------------------------------------------------------
\subsection{Discreization}\label{subsec:}
%-------------------------------------------------------------------------
%


%==========================================
\printbibliography[title=References Section~\thesection]
%==========================================



