% !TEX root = ../simfempy.tex
%
%==========================================
\section{Stokes problem}\label{sec:}
%==========================================
%
Let $\Omega\subset \R^d$, $d=2,3$ be the computational domain. We suppose to have a disjoined partition of its boundary:
$\partial\Omega=\GammaD\cup\GammaN$.
%
\begin{equation}\label{eq:stokes:strongform}
%
\left\{
\begin{aligned}
-\div\left(\mu\nabla v\right) + \nabla p = f \quad \mbox{in $\Omega$}\\
\div v  = g \quad \mbox{in $\Omega$}\\
v = \vD \quad \mbox{in $\GammaD$}\\
\mu\frac{\partial v}{\partial n} - p n = -\pN n\quad \mbox{in $\GammaN$}
\end{aligned}
\right.
%
\end{equation}
%
%
%-------------------------------------------------------------------------
\subsection{Weak formulation}\label{subsec:}
%-------------------------------------------------------------------------
%
Supposing $\abs{\GammaN}>0$, we have
%
\begin{equation}\label{eq:stokes:weakform}
%
\left\{
\begin{aligned}
&V :=  H^1(\Omega,\R^d)\quad Q := L^2(\Omega)\\
&(v,p)\in V\times Q:\quad a_{\Omega}(v,p; \phi\xi) + a_{\partial\Omega}(v,p; \phi,\xi) = l_{\Omega}(\phi,\xi)+l_{\partial\Omega}(\phi,\xi) \quad\forall (\phi,\xi)\in V\times Q\\
&\begin{aligned}
a_{\Omega}(v,p; \phi,\xi) &:= \int_{\Omega} \mu \nabla v:\nabla\phi - \int_{\Omega}p \div \phi + \int_{\Omega} \div v \xi\\
 a_{\partial\Omega}(v,p; \phi,\xi) &:= \int_{\GammaD}\frac{\gamma\mu}{h}v\cdot\phi - 
 \int_{\GammaD}\mu \left(  \dn{v}\cdot  \phi + v  \cdot \dn{\phi}\right) 
 + \int_{\GammaD} \left( p \phi_n - v_n \xi\right)\\
 l_{\Omega}(\phi,\xi) &:= \int_{\Omega} f\cdot\phi + \int_{\Omega} g\xi, \quad
 l_{\partial\Omega}(\phi,\xi) = \int_{\GammaD}\mu\vD\cdot\left( \frac{\gamma}{h}\phi - \dn{\phi}\right)- \int_{\GammaD}\vD_n\xi
 - \int_{\GammaN} \pN \phi_n.
\end{aligned}
\end{aligned}
\right.
%
\end{equation}
%
%---------------------------------------
\begin{lemma}\label{lemma:}
A regular solution of the formulation (\ref{eq:stokes:weakform}) satisfies (\ref{eq:stokes:strongform}).
\end{lemma}
%
%---------------------------------------
\begin{proof}
By integration by parts we have
%
\begin{align*}
a_{\Omega}(v,p; \phi,\xi) = \int_{\Omega} \left( -\mu \Delta v+\nabla p\right)\cdot\phi + 
\int_{\partial\Omega} \mu\dn{v}\cdot\phi - \int_{\partial\Omega} p\phi_n
+\int_{\Omega} \div v \xi
\end{align*}
%
and therefore with $a:=a_{\Omega}+a_{\partial\Omega}$ and $l:=l_{\Omega}+l_{\partial\Omega}$
%
\begin{align*}
a(v,p; \phi\xi) - l(v,p; \phi\xi) =& 
\int_{\Omega} \left( -\mu \Delta v+\nabla p -f \right)\cdot\phi + \int_{\Omega} (\div v -g)\xi
+ \int_{\GammaN} \left(\mu\dn{v}-p n + \pN n\right)\cdot\phi\\
&+  \int_{\GammaD}\frac{\gamma\mu}{h}v\cdot\phi - 
 \int_{\GammaD}\mu(v-\vD)\cdot\left( \frac{\gamma}{h}\phi - \dn{\phi}\right)
 - \int_{\GammaD} (v-\vD)\cdot n\xi
\end{align*}
%


\end{proof}
%



Alternatively, we can write the system as
%
\begin{equation}\label{eq:stokes:weakform2}
%
\left\{
\begin{aligned}
&(v,p)\in V\times Q:\quad a(v,p; \phi\xi) + b(v, \xi)- b(\phi, p) = l_{\Omega}(\phi,\xi)+l_{\partial\Omega}(\phi,\xi) \quad\forall (\phi,\xi)\in V\times Q\\
&\begin{aligned}
a(v,p; \phi,\xi) &:= \int_{\Omega} \mu \nabla v:\nabla\phi +\int_{\GammaD}\frac{\gamma\mu}{h}v\cdot\phi - 
 \int_{\GammaD}\mu \left(  \dn{v}\cdot n \phi + v n \cdot \dn{\phi}\right)\\
b(v, \xi) &:=  \int_{\Omega} \div v \xi - \int_{\GammaD} v_n \xi
\end{aligned}
\end{aligned}
\right.
%
\end{equation}
%



%
%-------------------------------------------------------------------------
\subsection{Implementations of Dirichlet condition}\label{subsec:}
%-------------------------------------------------------------------------
%
We write the discrete velocity space $V_h$ as a direct sum $V_h = \Vint_h \oplus \Vdir_h$, with $\Vdir_h$ corresponding to the discrete functions not vanishing on $\GammaD$. 
Splitting the matrix and right-hand side vector correspondingly, and letting $u^D_h\in \Vdir_h$ be an approximation of the Dirichlet data $\vD$ we have the traditional way to implement Dirichlet boundary conditions:
%
\begin{equation}\label{eq:StokesDirTrad}
\begin{bmatrix}
\bdryint{A} & 0 & -\transpose{\bdryint{B}}\\
0 & I & 0 \\
\bdryint{B} & 0 & 0
\end{bmatrix}
\begin{bmatrix}
\bdryint{v}_h\\
\bdrydir{v}_h\\
p_h
\end{bmatrix}
=
\begin{bmatrix}
\bdryint{f} - \Aintdir v^D_h\\
v^D_h\\
g - \bdrydir{B} v^D_h
\end{bmatrix}.
\end{equation}
%
As for the Poisson problem, we obtain an alternative formulation   
%
\begin{equation}\label{eq:StokesDirNew}
\begin{bmatrix}
\bdryint{A} & 0 & -\transpose{\bdryint{B}}\\
0 & \bdrydir{A} & 0 \\
\bdryint{B} & 0 & 0
\end{bmatrix}
\begin{bmatrix}
\bdryint{v}_h\\
\bdrydir{v}_h\\
p_h
\end{bmatrix}
=
\begin{bmatrix}
\bdryint{f} - \Aintdir v^D_h\\
\bdrydir{A} v^D_h\\
g - \bdrydir{B} v^D_h
\end{bmatrix}.
\end{equation}
%
%
%~~~~~~~~~~~~~~~~~~~~~~~~~~~~~
\subsubsection{Pressure mean}
%~~~~~~~~~~~~~~~~~~~~~~~~~~~~~
%
If all boundary conditions are Dirichlet, the pressure is only determined up to a constant. In order to impose the zero mean on the pressure, let $C$ the matrix of size $(1,nc)$
%
\begin{equation}\label{eq:StokesDirPmean}
\begin{bmatrix}
A  & -\transpose{B}& 0\\
B & 0 & \transpose{C}\\
0 & C & 0
\end{bmatrix}
\begin{bmatrix}
v\\
p\\
\lambda
\end{bmatrix}
=
\begin{bmatrix}
f\\
g\\
0
\end{bmatrix}.
\end{equation}
%
Let us considered solution of (\ref{eq:StokesDirPmean}) with $S=BA^{-1}\transpose{B}$, $T=CS^{-1}\transpose{C}$
%
\begin{equation}\label{eq:}
%
\left\{
\begin{aligned}
&A \tilde v &&= f\\
&S \tilde p &&= g-B\tilde v\\
&T \lambda &&= -C\tilde p\\
&S (p-\tilde p) &&= \transpose{C} \lambda\\
&A(v-\tilde v) &&=  \transpose{B} p
\end{aligned}
\right.
%
\end{equation}
%

%
%
%==========================================
\printbibliography[title=References Section~\thesection]
%==========================================
