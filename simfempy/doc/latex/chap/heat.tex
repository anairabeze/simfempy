% !TEX root = ../simfempy.tex
%
%==========================================
\section{Heat equation}\label{sec:Heat}
%==========================================
%
Let $\Omega\subset \R^d$, $d=2,3$ be the computational domain. We suppose to have a disjoined partition of its boundary:
$\partial\Omega=\GammaD\cup\GammaN\cup\GammaR$.
%
We consider the parabolic equation for the temperature $T$, heat flux $\vec{q}$ and heat release $\dot{q}$
%
\begin{yellow}[Heat equation (strong formulation)]
\begin{equation}\label{eq:HeatEquation}
%
\left\{
\begin{aligned}
\vec{q} = - \kdiff \nabla T\\
\rho C_p \dt{T} + \div\left(\vec{v} T\right)+\div\vec{q} = \dot{q} \quad \mbox{in $\Omega$}\\
T = \TD \quad \mbox{in $\GammaD$}\\
\kdiff\frac{\partial T}{\partial n} = \qN \quad \mbox{in $\GammaN$}\\
c_R T + \kdiff\frac{\partial T}{\partial n}= \qR \quad \mbox{in $\GammaR$}
\end{aligned}
\right.
%
\end{equation}
\end{yellow}
%
\begin{yellow}[Heat equation (weak formulation)]
Let $H^1_{f}:= \SetDef{u\in H^1(\Omega)}{\Rest{T}{\GammaD} = f}$. The standard weak formulation looks for 
$T\in H^1_{\TD}$ such that for all $\phi\in H^1_{0}(\Omega)$
%
\begin{equation}\label{eq:HeatEquationWeak}
\int_{\Omega}\rho C_p \dt{T}\phi - \int_{\Omega}\vec{v} T\cdot \nabla \phi
+  \int_{\Omega} \kdiff\nabla T \cdot\nabla \phi + \int_{\GammaR} c_R T \phi 
+ \int_{\GammaR\cup\GammaN} \vec{v}_n T \phi
= \int_{\Omega} \dot{q} \phi +  \int_{\GammaR} \qR \phi
\end{equation}
\end{yellow}
%
We can derive (\ref{eq:HeatEquationWeak}) from (\ref{eq:HeatEquation}) by the divergence theorem
%
\begin{align*}
\int_{\Omega} \div\vec{F} = \int_{\partial\Omega} \vec{F}_n\quad\overbrace{\Longrightarrow}^{F\to F\phi}\quad 
\int_{\Omega} (\div\vec{F}) \phi = -\int_{\Omega} \vec{F}\cdot \nabla\phi + \int_{\partial\Omega} \vec{F}_n\phi,
\end{align*}
%
which gives with $\vec{F}=\vec{v} + \vec{q}$%
\begin{align*}
\int_{\Omega} \div\left(\vec{v} + \vec{q}\right)\phi = -\int_{\Omega}\vec{v}\cdot\nabla\phi
+ \int_{\Omega}\kdiff\nabla T\cdot\nabla\phi+ \int_{\partial\Omega} \vec{F}_n\phi.
\end{align*}
%
Using that $\phi$ vanishes on $\GammaD$ we have
%
\begin{align*}
\int_{\partial\Omega} \vec{F}_n\phi =& \int_{\GammaN\cup\GammaR} \vec{F}_n\phi
= \int_{\GammaN\cup\GammaR} \vec{v}_n\phi + \int_{\GammaN\cup\GammaR} \vec{q}_n\phi,
\end{align*}
%
and then with the different boundary conditions, we find
%
\begin{align*}
\int_{\GammaN\cup\GammaR} \vec{q}_n\phi = \int_{\GammaD} \qN\phi + \int_{\GammaR}\left(\qR - c_R T\right)\phi
\end{align*}
%
%
%-------------------------------------------------------------------------
\subsection{Computation of the matrices for $\Pspace_h^1{\Omega}$}\label{subsec:}
%-------------------------------------------------------------------------
%
For the convection, we suppose that $\vec{v}\in \RT^0_h(\Omega)$ and let for given $K\in\Cells_h$ 
$\vec{v}=\sum_{k=1}^{d+1} v_k \Phi_k$. Using
%
\begin{align*}
x_k = x_{S_k}^K,\quad h_k = h_{S_k}^K, \quad \sigma_k = \sigma_{S_k}^K, n_k = n_{S_k}
\end{align*}
%
we compute
%
\begin{align*}
\int_K \lambda_j \vec{v}\cdot \nabla \lambda_i = \sum_{k=1}^{d+1} v_k \int_K \lambda_j \Phi_k\cdot \nabla \lambda_i\\
\int_K \lambda_j \Phi_k\cdot \nabla \lambda_i = -\frac{\sigma_k \sigma_i}{h_kh_i} \int_K  \lambda_j (x-x_k) \cdot n_i
= -\frac{\sigma_k \sigma_i}{h_kh_i} \sum_{l=1}^{d+1} (x_l-x_k) \cdot n_i \int_K  \lambda_j\lambda_l 
\end{align*}
%




%==========================================
\printbibliography[title=References Section~\thesection]
%==========================================



