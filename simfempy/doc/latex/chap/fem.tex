% !TEX root = ../simfempy.tex
%
%==========================================
\section{Finite elements on simplices}\label{sec:}
%==========================================
%
%
%-------------------------------------------------------------------------
\subsection{Simplices}\label{subsec:}
%-------------------------------------------------------------------------
%
We consider an arbitrary non-degenerate simplex $K=(x_0,x_1,\ldots, x_{d})$. The (signed) volume of $K$ is given by
%
\begin{equation}\label{eq:}
|K| = \frac{1}{d!} \det(x_1-x_{0},\ldots, x_{d}-x_{0})= \frac{1}{d!} \det(1,x_{0},x_1\ldots, x_{d})\quad 1=\transpose{(1,\ldots,1)}.
\end{equation}
%
The $d+1$ sides $S_k$ (co-dimension one, $d-1$-simplices or facets) are defined by
$S_k=(x_0,\ldots, \cancel{x_k}, \ldots, x_{d})$. The height is $h_k=|P_{S_k}x_k - x_k|$, where $P_S$ is the orthogonal projection on the hyperplane associated to $S_k$. We have
%
\begin{align*}
h_k = d\frac{|K|}{|S_k|} \qquad\mbox{(and for $d=3 \; |S_k| = \frac12 |u\times v| $)}
\end{align*}
%
%
%-------------------------------------------------------------------------
\subsection{Barycentric coordinates}\label{subsec:}
%-------------------------------------------------------------------------
%
%
Any polynomial in the barycentric coordinates can be integrated exactly.
%
\begin{yellow}[Integration on $K$]
\begin{equation}\label{eq:}
\int_K \prod_{i=1}^{d+1}\lambda_i^{n_i} \,dv = d!|K|\frac{\prod\limits_{i=1}^{d+1} n_i!}{\left( \sum\limits_{i=1}^{d+1} n_i + d\right)!}
\end{equation}
\end{yellow}
%
see \cite{EisenbergMalvern73}, \cite{VermolenSegal18}.
%
%
\begin{yellow}[Gradient of $\lambda_i$]
\begin{align*}
\nabla \lambda_i = - \frac{1}{h_i}\vec{n_i}. 
\end{align*}
\end{yellow}
%
%
%-------------------------------------------------------------------------
\subsection{Finite elements}\label{subsec:}
%-------------------------------------------------------------------------
%
%
We consider a family $\allmeshes$ of regular simplicial meshes $h$ on a polyhedral domain $\Omega\subset \R^d$. 
The set of simplices of $h\in\allmeshes$ is denoted by $\Cells_h$, and its $d-1$-dimensional sides by $\Sides_h$, divided into interior and boundary sides $\SidesInt_h$ and $\SidesBdry_h$, respectively. 
The set of $d+1$ sides of $K\in\Cells_h$ is $\Sides_h(K)$. To any side $S\in\Sides_h$ we associate a unit normal vector $n_S$, which coincides with the unit outward normal vector $n_{\partial\Omega}$ if $S\in\SidesBdry_h$.

For $K\in\Cells_h$ and $S\in\Sides_h$, or $S\in\Sides_h(K)$ we denote
%
\begin{align*}
x_K:\quad& \mbox{barycenter of $K$} \quad& x_S:\quad & \mbox{barycenter of $S$}\\
x_S^K:\quad & \mbox{vertex opposite to $S$ in $K$} \quad& h_S^K:\quad & \mbox{distance of $x_S^K$ to $S$}\\
\sigma_{S}^K :=&\begin{cases}
+1 & \mbox{if $n_S=n_K$},\\
-1 & \mbox{if $n_S=-n_K$}.
\end{cases}\quad&
%\rho_K := \frac{x - x_K^*}{d},\quad &\Rest{\rho_h}{K} = \rho_K
\lambda_S^K:\quad& \mbox{barycentric coordinates of $K$} 
\end{align*}

For $k\in\N_0$ we denote by $\Cspace_h^k(\Omega)$ the space of piecewise $k$-times differential functions with respect to $\Cells_h$, and 
piecewise differential operators $\nabla_h:\Cspace_h^l(\Omega)\to \Cspace_h^{l-1}(\Omega,\R^d)$ ($l\in\N$) by 
$\Rest{\nabla_h q}{K} := \nabla\left(\Rest{q}{K}\right)$ for $q\in \Cspace_h^l(\Omega)$ and similarly for 
$\div_h: \Cspace_h^{l}(\Omega,\R^d)\to \Cspace_h^{l-1}(\Omega)$.
We frequently use the piecewise Stokes formula
%
\begin{equation}\label{eq:StokesPiecewise}
\int_{\Omega} \nabla_h q v + \int_{\Omega} q \div_h v = \int_{\SidesInt_h}\jump{q v_n} +  \int_{\SidesBdry_h}q v_n,
\end{equation}
%
where $\int_{\Sides_h}=\sum_{S\in\Sides_h}\int_S$ and $n$ in the sum stands for $n_S$.

The subspace of piecewise polynomial functions of order $k\in\N_0$ in $C^k_h(\Omega)$ is denoted by $\Dspace_h^k(\Omega)$ and the $L^2(\Omega)$-projection by $\pi_h^k:L^2(\Omega)\to\Dspace_h^k(\Omega)$.

The subspaces of continuous and nonconforming piecewise-polynmial functions are for $k\in\N$
%
\begin{equation}\label{eq:}
\Pspace_h^k(\Omega) := \Dspace_h^k(\Omega)\cap C(\overline{\Omega}),\quad
\CR_h^k(\Omega) := \SetDef{q \in \Dspace_h^k(\Omega)}{\int_S\jump{q}p=0\;\forall S\in\SidesInt_h,\forall p\in P^{k-1}(S)}.
\end{equation}
%  
The standard finite element theory provides canonical basis functions and interpolation operators, which however need special care in the nonconforming case for even $k$. 


The restrictions of the basis functions of $\Pspace^1_h(\Omega)$ to the simplex $K$ are the barycentric coordinates $\lambda_S^K$ associated to the node opposite to $S$ in $K$.
%
\begin{yellow}[Formulae for $\Pspace^1_h$]
\begin{equation}\label{eq:}
\nabla \lambda_S^K = -\frac{\sigma_{S}^K}{h_S^K}n_S,\quad \frac{1}{|K|}\int_K \lambda_S^K = \frac{1}{d+1}.
\end{equation}
\end{yellow}%
%

Denote in addition the basis of $\CR^1_h(\Omega)$ by $\psi_S$, we have
%
\begin{yellow}[Formulae for $\CR^1_h$]
\begin{equation}\label{eq:}
\Rest{\psi_S}{K} = 1- d \lambda_S^K, \quad \nabla \Rest{\psi_S}{K} =
\frac{|S|\sigma_{S}^K}{|K|}n_S,\quad \frac{1}{|K|}\int_K \psi_S = \frac{1}{d+1}.
\end{equation}
\end{yellow}%
%

The Raviart-Thomas space for $k\ge0$ is given by
%
\begin{equation}\label{eq:}
\RT^k_h(\Omega):= \SetDef{v\in D^k_h(\Omega,\R^d)\oplus X^k_h}{\int_S\jump{v_n}p=0\;\forall S\in\SidesInt_h,\forall p\in P^{k}(S)}
\end{equation}
%
where $X^k_h:=\SetDef{xp}{\Rest{p}{K}\in \Phom(K)\; \forall K\in\Cells_h}$ with $\Phom(K)$ the space of $k$-th order homogenous polynomials.

Then the Raviart-Thomas basis function of lowest order is given by
%
\begin{yellow}[Formulae for $\RT^0$]
\begin{equation}\label{eq:BasisFctRt}
\Rest{\Phi_S}{K} := \sigma_{S}^K \frac{x-x_S^K}{h_S^K},\quad 
\int_K \div \Rest{\Phi_S}{K} = \sigma_{S}^K \frac{d|K|}{h_S^K}=  \sigma_{S}^K|S|,\quad \frac{1}{|K|}\int_K \Phi_S = \sigma_{S}^K\frac{x_K^* - x_S^K}{h_S^K}.
\end{equation}
\end{yellow}%

%

%==========================================
\printbibliography[title=References Section~\thesection]
%==========================================

